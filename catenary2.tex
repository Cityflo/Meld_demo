\documentclass[11pt]{amsart}

\begin{document}

In physics and geometry, a \textit{catenary} is the curve that an idealised hanging chain or cable assumes under its own weight when supported only at its ends. The curve has a U-like shape, superficially similar in appearance to a parabola, but it is not a parabola: it is a (scaled, rotated) graph of the hyperbolic cosine. 

Mathematically, the catenary curve is the graph of the hyperbolic cosine function, and in Cartesian coordinates has the form:
\begin{equation}
y = \frac{a \: (e^{\frac{x}{a}} + e^{-\frac{x}{a}})}{2}
\end{equation}

What makes the catenary arch important is its ability to withstand weight. For an arch of uniform density and thickness, supporting only its own weight, the catenary is the ideal curve.

\end{document}  