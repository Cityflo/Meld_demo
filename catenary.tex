\documentclass[11pt]{amsart}

\begin{document}

In physics and geometry, a \textit{catenary} is the curve that an idealised hanging chain or cable assumes under its own weight when supported only at its ends. The curve has a U-like shape, superficially similar in appearance to a parabola, but it is not a parabola: it is a (scaled, rotated) graph of the hyperbolic cosine. 

Mathematically, the catenary curve is the graph of the hyperbolic cosine function, and in Cartesian coordinates has the form:
\begin{equation}
y = a \: \mathrm{cosh}(\frac{x}{a})
\end{equation}

It is often said that Galileo thought the curve of a hanging chain was parabolic. In his \textit{Two New Sciences (1638)}, Galileo says that a hanging cord is an approximate parabola, and he correctly observes that this approximation improves as the curvature gets smaller and is almost exact when the elevation is less than 45 degrees. That the curve followed by a chain is not a parabola was proven by Joachim Jungius (1587?1657).

What makes the catenary arch important is its ability to withstand weight. For an arch of uniform density and thickness, supporting only its own weight, the catenary is the ideal curve.


\end{document}  
